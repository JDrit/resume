\documentclass[a4paper,margin,line]{resume}
\usepackage[defblank]{paralist}
\usepackage{pdfpages}
\usepackage{anysize}
\usepackage[unicode]{hyperref}
\hypersetup{
	pdftitle={Joseph Batchik's Resume},
	pdfauthor={Joseph Batchik},
	pdfborder={0 0 0},
	unicode=true
}
\marginsize{0.375in}{1.875in}{0.375in}{0.375in}
\setdefaultitem{\footnotesize \textbullet}{}{}{}{}{}
\setdefaultleftmargin{0em}{}{}{}{}{}
\setdefaultenum{(a)}{(1)}{}{}{}{}
\newcommand{\rurl}[1]{\hfill {\footnotesize \url{#1}}}
\newcommand{\rdate}[1]{\hfill {\small #1}}
\renewcommand{\employer}[5]{\item[#1] - #2 \rdate{#3} \\* #4 \rurl{#5} \\*}
\begin{document}
\name{\Large Joseph D. Batchik}
\begin{resume}
\section{\mysidestyle Contact \\ Information} \vspace{2mm}
	\begin{asparablank}
    \item 6445 Sundown Trail  \hfill (410) 599-3550
		\item Columbia MD, 21044 \hfill \href{http://jd.batchik.net/}{http://jd.batchik.net}
		\item  \hfill \href{mailto:josephbatchik@gmail.com}{josephbatchik@gmail.com}
	\end{asparablank}

\section{\mysidestyle Objective}
	To secure a co-op position for the summer of 2015.

\section{\mysidestyle Experience}
	\begin{asparadesc}
        \employer{Google}{New York, NY}{Summer 2014}{SRE Engineering Practicum Intern}
        {http://google.com/}
		\small
        Worked on load testing infrastructure and implemented performance increases 
        in multiple systems as a site reliability engineer. Languages / tools used: 
        Java, Python, various Google data stores.
		\normalsize
		\\
		\employer{Amazon}{Seattle, WA}{Spring 2014}{Software Developer Engineer Intern}
        {http://amazon.com/}
		\small
	    Worked with various Amazon cloud products such as Cloud-Search, SNS, and SQS 
        to develop an internal search tool for the Enterprise Data Warehouse team. 
        Languages / tools used: Java, various AWS products
        \normalsize
		\\
		\employer{John Hopkins University Applied Physics Lab}{Laurel, MD}
        {Summer 2013}{Engineering Intern}{http://jhuapl.edu/}
		\small
		Implemented a sensor management system used to control and collect data from 
        multiple telescopes remotely. Languages / tools used: Java, ant, svn, SQL, 
        Google Protocol Buffers.
        \normalsize
		\end{asparadesc}

\section{\mysidestyle Education}
	\begin{compactdesc}
		\item[Rochester Institute of Technology] - Rochester, NY \rdate{September 2012 - Present}
		\begin{compactitem} { \small
			\item Major: Computer Science, Minor: American Politics
            \item Dean's List (3.6 GPA)
			\item Expected graduation: May 2016
		} \end{compactitem}
	\end{compactdesc}

\section{\mysidestyle Technical Skills \& Certifications}
	\begin{compactdesc}
		\item[Languages] \begin{inparaenum} { \small
			\item Python
			\item Java
			\item HTML / CSS
			\item Ruby
			\item Go
		} \end{inparaenum}
        \item[Certifications] \begin{inparaenum} { \small
            \item Cloudera Certified Developer for Apache Hadoop, 2012    
        } \end{inparaenum}
		\item[Tools] \begin{inparaenum} { \small
			\item git
            \item svn
            \item vim
            \item rails
            \item postgreSQL
            \item Google Protocol Buffers
		} \end{inparaenum}
	\end{compactdesc}

\section{\mysidestyle Self-Directed Projects}
    \begin{asparablank}
        \item \textbf{Sys Mon}: Developed a system monitoring tool in Ruby to monitor 
            load average, memory usage, and IO. All the log 
            data is viewable through a web interface with graphs over time.
        \\
        \item \textbf{Github Stats}: Create a data anaylzer in Go to determine the language
            usage of repositories on GitHub. This showed the connections between
            various languages and usage over time. This data anaylzer was set up in a
            distributed manner using RabbitMQ.
        \\
        \item \textbf{Mobile News}: Wrote an Android application to allow users to read 
            and edit the Computer Science House’s internal news network on a 
            native application.
        \\
        \item \textbf{Housing Site}: Designed a Python web server to allow for room 
            registration for future room assignments for members of the Computer
            Science House.
        \\
        \item \textbf{LDAP Profiles}: Created a Ruby on Rails web server that is a web 
            interface to LDAP servers. It uses SASL authentication with WebAuth 
            to provide security.
    \end{asparablank}
\section{\mysidestyle Awards}
    \begin{asparablank}
        \item \textbf{Maryland Math Engineering Science Achievement}: Won 1st place
            in both regional and state level competitions for developing a maze
            traversal algorithm in Python.
        \\
        \item \textbf{Website Excellence Award, FIRST Robotics}: Developed and maintained an
            award winning website for my school's FIRST robotics team.
    \end{asparablank}
\section{\mysidestyle Clubs \& Activities}
    \begin{asparablank}
        \item \textbf{CSH (Computer Science House)}: An organization at RIT that 
            provides a living and learning environment with access to unique facilities
            and hands-on learning, all in a social environment.
        \\
        \item \textbf{Robotics Team 2537}: Was part of a robotics in which I built and
            maintained the team's award winning website.
    \end{asparablank}

\end{resume}
\end{document}
